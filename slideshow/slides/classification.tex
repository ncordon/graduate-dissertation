\subsection{Classification of states}
 \begin{frame}\frametitle{Communicating classes}

    \begin{definition}
    The state \(i\in S\) communicates with \(j\in S\), and we will write it like 
    \(i \rightarrow j\) iff \(p_{i,j}^{(n)} > 0\) for some \(n > 0\). It is a 
    transitive relation.

    If \(i \goto j\) and \(j \goto i\) then we will say that both states communicate and
    we will represent it as \(i \dgoto j\).
    \end{definition}

    \begin{definition}
    State \(i\in S\) is essential if \(i \goto j\) always implies \(j \goto i\). We denote the set
    of essential states by \(S^{\ast}\).
    \end{definition}

    \begin{lemma}
    If \(i\in S^{\ast}\) and \(i \goto j\), then \(j \in S^{\ast}\)
    \end{lemma}

    \begin{proposition}
    \(\dgoto\) is an equivalence relation in \(S^{\ast}\), and we can decompose:

    \[
    S^{\ast} = C_1 \cup C_2 \cup \ldots \cup C_r
    \]

    where for every \(i, j \in C_k\), it holds \(i \dgoto j\).
    \end{proposition}
 \end{frame}

 \begin{frame} \frametitle{Canonic form}
    \vspace{2em}
    For a DTMC, we can reorder the states of the matrix: let $P_k$ be the transition matrix for 
    the states in $C_k$ (which only communicate with themselves); $Q_k$ the probabilities of
    going from $S - S^{\ast}$ to $C_k$ (and never returning); and $R$ the probability
    of the non-essential states communicating with themselves. If on top of that we consider:
    
    \begin{enumerate}
    \item States are ordered inside each class, \(S - S^{\ast}\) included.
    \item \(C_k = [s_k]\) where \(s_k\) is the minimum of \(C_k\) and \(s_{i} < s_{i + 1}\) for all \(i\).
    \end{enumerate}
    
    then the transition (stochastic by rows) matrix we get is:
 
  \[
  \left(\begin{array}{cccccccccc}
  \gcell P_1    &        &            &       &        &       &            &        & \\
                & \rcell & \rcell     & \rcell&        &       &            &        & \\
                & \rcell & \rcell P_2 & \rcell&        &       &            &        & \\
                & \rcell & \rcell     & \rcell&        &       &            &        & \\
                &        &            &       & \ddots &       &            &        & \\
                &        &            &       &        &\bcell &   \bcell   & \bcell & \\
                &        &            &       &        &\bcell & \bcell P_r & \bcell & \\
                &        &            &       &        &\bcell &   \bcell   & \bcell & \\
    \rowcolor{gray!20}
      Q_1       &        &   Q_2      &       & \ldots &       &    Q_r     &        & R\\
  \end{array}\right)
  \]
 \end{frame}

 \begin{frame} \frametitle{Recurrence and transcience}
    \begin{definition}
    An state \(i\in S\) is called recurrent iff \(f_{i,i} = 1\) and transient iff \(f_{i,i} < 0\)
    \end{definition}

    \begin{proposition}
    If \(i\) is a recurrent state and \(i\) communicates with \(j\), then 
    \(f_{j,i} = f_{i,j} = f_{j,j} = 1\), and \(j\) is recurrent. Therefore, all the states of a
    communicating class are either recurrent or transient.
    \end{proposition}
    
    \begin{definition}
    We will say a communicating class is recurrent (resp. transient) iff all of its states
    are recurrent (resp. transient). We will say a communicating class \(C\) is closed if it
    holds \(i \goto j\) and \(i\in C\) implies \(j\in C\).
    \end{definition}
    
    \begin{proposition}
    All the closed irreducible subchains of a DTMC are recurrent, and every DTMC has
    at least one subchain of such type. A state is transient iff it is non-essential.
    \end{proposition}
  
    \begin{corollary}
    In a DTMC, a communicating class is recurrent iff it is closed.
    \end{corollary}
    
 \end{frame}
 
 \begin{frame} \frametitle{Absorption}
    \vspace{2em}
    
    \begin{definition}
    A state \(i\in S\) is absorbing iff \(p_{i,i} = 1\).
    \end{definition}
  
  \begin{definition}
    \begin{enumerate}
    \item The absorbing time as \(\tau^{\ast} = \min \{n > 0: X_n \in S^{\ast}\}\).
    \item The absorbing probability as \(f_{i}^{\ast} = \probi{\tau^{\ast} < \infty}\).
    \item Taking \(j\in S^{\ast}\), we can define the probability of being absorbed by \(j\) as \(f_{i,j}^{\ast} = \probi{\tau^{\ast} < \infty, X_{\tau^{\ast}} = j}\)
    \item Since we can decompose \(S^{\ast} = C_1 \cup C_2 \cup \ldots \cup C_r\), given \(C = C_i\), we can define the probability of the state \(i\) being absorbed by the subchain \(C\) as:
    \end{enumerate}

    \[
    f_{i,C}^{\ast} = \probi{\tau^{\ast} < \infty, X_{\tau^{\ast}} \in C}
    \]
    \end{definition}

    \begin{proposition}
    The following holds for each \(i \not\in S^{\ast}, j\in S^{\ast}\):

    \begin{enumerate}
    \item \(f_i^{\ast} = \sum_{j \in S^{\ast}} p_{i,j} + \sum_{j \not\in S^{\ast}} p_{i,j} f_{j}^{\ast}\)
    \item \(f_{i,j}^{\ast} = p_{i,j} + \sum_{k \not\in S^{\ast}} p_{i,j} f_{k,j}^{\ast}\)
    \item \(f_{i,C}^{\ast} = \sum_{j \in C} p_{i,j} + \sum_{j \not\in S^{\ast}} p_{i,j} f_{j}^{\ast}\)
    \end{enumerate}
    \end{proposition}
 \end{frame}
 
 \begin{frame} \frametitle{The fundamental matrix}
    \vspace{2em}
    
    \begin{proposition}
    The following relations hold:

    \[
    F^{\ast} = (I + V)Q \qquad
    f^{\ast} = (I + V)q
    \]

    The distribution for \(\tau^{\ast}\) is \(\probi{\tau^{\ast} = n} = R^{n - 1}q, \quad n \in \mathbb{N}\)
    \end{proposition}
  
    \begin{definition}
    The matrix \(N := (I - R)^{-1}\) is called fundamental matrix for the Markov chain. 
    The entry \(n_{ij}\) represents the number of times the process is in transient state \(j\) 
    if it is started in also transient state \(i\).
    \end{definition}

    \begin{definition}
    We will call \(t_i\) to the mean number of steps before the chain is absorbed, given that 
    the chain starts in \(i\), with \(i\) a transient state.
    \end{definition}
    
    \begin{proposition}
    If \(S - S^{\ast}\) is finite, then \(N = (I - R)^{-1}\) exists, \(N = I + V\) and:

    \[
    F^{\ast} = N\cdot Q \qquad 
    t = N \cdot \onecol \qquad
    f^{\ast} = 1
    \]

    where the last equation tells us the absorption would take place almost surely.
    \end{proposition}
  
 \end{frame}



