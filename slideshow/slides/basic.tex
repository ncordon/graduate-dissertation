\subsection{Time homogeneous Markov chains}
 \begin{frame} \frametitle{Time homogeneous Markov chains}
    \begin{columns}
    \begin{column}{0.5\textwidth}
        \begin{definition}
        Given a Markov chain \(\{X_n\}\) with finite space state \(S\), it is said to be a 
        time-homogeneous Markov chain (and we will note it DTMC) iff the transition 
        probabilities do not depend on time:

        \[
        \forall i, j, n \quad P(X_{n + 1} = j | X_{n} = i) = p_{i, j}
        \]

        \(p_{i, j}\) is called one-step transition probability, \(P = (p_{i, j})_{i, j \in S}\) the transition matrix.
        \end{definition}
        
        \begin{remark}
        There is a bijection between a DTMC with $S = \{1, \ldots, m\}$,
        a by-row stochastic matrix $P$ of dimension $m \times m$ (we will note it 
        $P_{m\times m}$) and a directed graph $G(P):=(S, E, P)$ where there is an
        edge $e = (i, j), e \in E$ iff $p_{i,j} > 0$
        \end{remark}
    \end{column}
    
    \begin{column}{0.5\textwidth}
        \begin{definition}
        We define the hitting time for the state \(j \in S\) as 
        $\tau_j = min \{n > 0 : X_n = j \}$
        \end{definition}
        
        \begin{definition}
        The probability that the first hitting time for $j$, starting at $i$, is $n$, is:

        \[
        f_{i,j}^{(n)} = P_i\left(\tau_j = n\right)
        \]

        and we will define the (first) hitting probability for \(j\) starting at \(i\) as:

        \[
        f_{i, j} = P_i\left(\tau_j < \infty\right)
        \]
        \end{definition}
        
        The values $f_{i, j}$ verify the recurrence:

        \[
        f_{i,j} = p_{i,j} + \sum_{k\neq j} p_{i,k} f_{k,j}
        \]
    \end{column}
    \end{columns}
 \end{frame}
