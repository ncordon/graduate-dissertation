\subsection{Introducción a la probabilidad}
 \begin{frame}\frametitle{Definition}
    \vspace{1em}
    \begin{definition}
    A stochastic random process \(\{X_n\}_{n\in \naturals}\), is a Markov chain iff the 
    future process \(\{X_k\}_{k > n}\) is independent of the past process 
    \(\{X_k\}_{k < n}\) conditionally on \(X_n\).
    \end{definition}

    \begin{lemma}
    \(\{X_n\}_{n \ge 0}\) is a Markov chain iff for all \(n\in\mathbb{N}\) and 
    arbitrary \(s_0, \ldots, s_{n + 1} \in S\) it holds:

    \[
    P(X_{n + 1} = s_{n + 1} | X_n = s_n, \ldots, X_0 = s_0) = P(X_{n + 1} = s_{n + 1} | X_n = s_n)
    \]
    \end{lemma}
    
    \begin{definition}
    Let \(S\) be the state space. Then we define the transition probabilities as:

    \begin{align*}
    & p_{i, j}(k, n) = P(X_n = j | X_k = i),\quad i,j \in S, \quad n,k \in \mathbb{N}, n\neq k \\
    & p_{i, j}(n, n) = \mathds{1}_{i = j}
    \end{align*}
    \end{definition}
    
    \begin{proposition}[Chapman-Kolmogorov equation]
    \(P(k, n) = (p_{i, j}(k, n))_{i,j \in S}\) holds:
    \(P(k, n) = P(k, l) \cdot P(l, n), \quad \forall k \le l \le n
    \)
    \end{proposition}

 \end{frame}
 
 \begin{frame} \frametitle{Time homogeneous Markov chains}
    \begin{definition}
    Given a Markov chain \(\{X_n\}\) with finite space state \(S\), it is said to be a 
    time-homogeneous Markov chain (and we will note it DTMC) iff the transition 
    probabilities do not depend on time:

    \[
    \forall i, j\in S, \,\forall n \in \mathbb{N} \qquad p_{i, j} (n, n + 1) = p_{i, j} \in [0, 1]
    \]

    \(p_{i, j}\) is called one-step transition probability and \(P = (p_{i, j})_{i, j \in S}\) the transition matrix.
    \end{definition}

    \begin{proposition}
    Given \(\{X_n\}\) a DTMC it holds \(P(m, n) = P^{n - m}\)
    \end{proposition}
 \end{frame}
