 \begin{frame} \frametitle{Recurrence and transcience}
    \vspace{2em}
    \begin{definition}
    An state \(i\in S\) is called recurrent iff \(f_{i,i} = 1\) and transient iff \(f_{i,i} < 1\)
    \end{definition}

    \begin{proposition}
    If \(i\) is a recurrent state and \(i\) communicates with \(j\), then
    \(f_{j,i} = f_{i,j} = f_{j,j} = 1\), and \(j\) is recurrent. Therefore, all the states of a
    communicating class are either recurrent or transient.
    \end{proposition}

    \begin{definition}
    We will say a communicating class is recurrent (resp. transient) iff all of its states
    are recurrent (resp. transient). We will say a communicating class \(C\) is closed if it
    holds \(i \goto j\) and \(i\in C\) implies \(j\in C\).
    \end{definition}

    \begin{proposition}
    All closed irreducible subchains of a DTMC are recurrent, and every DTMC has
    at least one subchain of such type. A state is transient iff it is non-essential.
    \end{proposition}

    \begin{corollary}
    In a DTMC, a communicating class is recurrent iff it is closed.
    \end{corollary}

 \end{frame}