\subsection{Mean first passage time}
 \begin{frame}\frametitle{Steady states}
    \vspace{2em}
    
    \begin{definition}
    We call the expected number of steps to reach a state \(j\) from initial state 
    \(i\) as mean first passage time \(m_{ij}\).
    
    We define the mean recurrence time for \(i\) as
    \(r_{i} := \expecti{\tau_{i}} = \sum_{n = 1}^{\infty} n f_{ii}^{(n)}\)
    \end{definition}
    
    \begin{proposition}
    Defining \(M = (m_{ij})\), \(D = (\delta_{ij} \cdot r_i)\), \(C = (1) \), then 
    it holds:
    
    \[
    M = PM + C - D
    \]
    \end{proposition}
    
    \begin{proposition}
    Given an ergodic Markov chain, then \(r_i = w_i^{-1}\) where \(w\) is its unique steady state.
    \end{proposition}

    \begin{theorem}
    Given \(P\) an ergodic matrix, \(w\) its steady state, \(Z = (I - P + W)^{-1} = (z_{ij})\) and \(M = (m_{ij})\), it holds:

    \[ m_{ij} = \frac{z_{jj} - z_{ij}}{w_j} \]
    \end{theorem}
    
\end{frame}
