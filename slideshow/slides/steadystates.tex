\subsection{Steady states}
 \begin{frame}\frametitle{Steady states}
    \vspace{2em}
    \begin{proposition}
    Given \(P_{m\times m}\) an stochastic matrix, then every steady state \(v\) assigns 
    \(0\) to the transient states, i.e. if \(i\) -th state is transient, then 
    \((vP)_i = 0\)
    
    ~\\
    \label{prop:ss-transient-zero}
    \end{proposition}
    Each matrix \(P_1, \ldots, P_r\) from its canonic form is irreducible. Therefore, 
    by the fundamental theorem, each \(P_i\) has a unique steady vector \(v^{(i)}\). 
    If \(P_i\) corresponds to the states \([l_i, l_i + 1 \ldots, u_i]\). We are going 
    to call:
    
    \[
    \widetilde{v}^{(i)}_j = \twopartdef{0}{j < l_i \textrm{ or } j > u_i}{v^{(i)}_k}{j = l_{i + k - 1}}
    \]

    \begin{proposition}
    Given an stochastic matrix \(P\) written in its canonic form, then its space of 
    steady states is a convex hull given by:

    \[
    \bigg\{\sum_{i = 1}^r \alpha_i \widetilde{v}^{(i)}: 0 \le \alpha_i, \sum_{i = 1}^{r} \alpha_i = 1\bigg\}
    \]
    \end{proposition}
\end{frame}
