\begin{frame}\frametitle{Motivation}
 \begin{columns}
 \begin{column}{0.5\textwidth}
  \begin{itemize}
   \item Highly researched area of mathematics and computer science. Used in 
   wide variety of fields: weather prediction, actuarial science, physics,
   speech recognition, genetics, \ldots
   \item R is probably the most used language among statisticians and the main 
   competitor of python in machine learning programming. The R package 
   \href{https://github.com/spedygiorgio/markovchain}{markovchain} is an 
   open source library which aims to provide easy representations, plotting 
   and analysis of discrete time Markov chains.
  \end{itemize}
 \end{column}
 
 \begin{column}{0.4\textwidth}
   \begin{figure}[htbp]
  \centering
  \includesvg[width=\textwidth]{../imgs/percentile}
  \end{figure}
 \end{column}
\end{columns}
 
 \begin{columns}
  \begin{column}{\textwidth}
  \begin{itemize}
  \item Lots of methods could be improved from the mathematical correctness perspective
   and the computational efficiency one (use fewer iterative methods and more algebraic 
   ones, improve efficiency for methods in general). High impact of the contributions.
  \end{itemize}
  \end{column}
 \end{columns}
\end{frame}

\begin{frame}[fragile] \frametitle{Maintainability?}
\vspace{2.7em}
\begin{minted}[fontsize=\tiny]{C++}
List communicatingClasses(S4 object) {
  NumericMatrix matr = object.slot("transitionMatrix");
  List temp = commclassesKernel(matr);
  LogicalMatrix adjMatr = temp["C"];
  int len = adjMatr.nrow();
  List classesList;
  CharacterVector rnames = rownames(adjMatr);
  for(int i = 0; i < len; i ++) {
    bool isNull = false;
    LogicalVector row2Check = adjMatr(i, _);
    CharacterVector proposedCommClass;
    for(int j = 0; j < row2Check.size(); j++) {
      if(row2Check[j] == true) {
        String rname = rnames[j];
        proposedCommClass.push_back(rname);
      }
    }
    if (i > 0) {
      for(int j = 0; j < classesList.size(); j ++) {
        bool check = false;        
        CharacterVector cv = classesList[j];
        std::set<std::string> s1, s2;
        for(int k = 0; k < cv.size(); k ++) {
          s1.insert(as<std::string>(cv[k]));
          if(proposedCommClass.size() > k)
            s2.insert(as<std::string>(proposedCommClass[k]));
        }
        check = std::equal(s1.begin(), s1.end(), s2.begin());
        if(check) {
          isNull = true;
          break;
        }
      }
    }
    if(!isNull) 
      classesList.push_back(proposedCommClass);
  }
  return classesList;
}
\end{minted}
\end{frame}

\begin{frame}[fragile] \frametitle{Correctness}
\begin{algorithm}[H]
\begin{algorithmic}[1]
  \REQUIRE $P_{m \times m}$ an stochastic matrix by columns
  \STATE{Compute $v = (e_1, \ldots, e_k)$, eigen values for $P$, i.e.
    \[
      \forall e_j \in \mathbb{C} \quad \exists v_j =
              \left(\begin{array}{c}
                v_{j1} \\
                \vdots \\
                v_{jm}
              \end{array}\right)
      \in \mathbb{C}^m: Pv_j = e_j v_j
    \]
  }
  \STATE{Compute $S = (Re(v_1) | \ldots | Re(v_k)) = (u_1| \ldots |u_k)$, eigen vectors for eigen values $1$}
  \FOR{$i = 1, \ldots, k$}
    \STATE{$s_i = \sum_{j = 1}^m u_{ij}$}
    \STATE{Normalize each column $u_i := u_i / s_i$}
  \ENDFOR
  \NEWLINE
  \RETURN{$S = (u_1 | \ldots | u_k)$}
\end{algorithmic}
\caption{Former \texttt{steadyStates} algorithm}
\label{alg:computeSteadyStatesFirst}
\end{algorithm}

\end{frame}

\begin{frame} \frametitle{Correctness}
\begin{itemize}
 \item Algorithms to compute eigen vectors are iterative and can take up to 
 $\mathcal{O}(m^3)$ or $\mathcal{O}(m^4)$ (in each iteration!). Also, because of 
 the use of convergence, we could end up having rounding problems (i.e. how 
 to securely search for the $1$ eigen values) and uncertainty about the 
 algorithmic complexity (we do not know how many iterations we need for them to converge beforehand).
 \item Taking real parts does not ensure we will end up with linearly independent eigen vectors. If the matrix $P$ has real eigen vectors $\{u, v\}$ which are linearly independent and are tied to the eigen value $1$, then $\{u + iu, u + iv\}$ are also independent eigen vectors whose real parts are $\{u,u\}$. 
\end{itemize}

\bigskip
As a result, we could get fewer steady states than we should.
\end{frame}


