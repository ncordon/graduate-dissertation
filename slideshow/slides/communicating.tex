\subsection{Classification of states}
 \begin{frame}\frametitle{Communicating classes, canonic form}
    \vspace{2em}
    \begin{columns}
        \begin{column}{0.5\textwidth}
            \begin{definition}
            The state \(i\in S\) communicates with \(j\in S\), and we will write it like 
            \(i \rightarrow j\) iff \(p_{i,j}^{(n)} > 0\) for some \(n > 0\). It is a 
            transitive relation.

            If \(i \goto j\) and \(j \goto i\) then we will say that both states communicate and
            we will represent it as \(i \dgoto j\).
            \end{definition}

            \begin{definition}
            State \(i\in S\) is essential if \(i \goto j\) always implies \(j \goto i\). We denote the set
            of essential states by \(S^{\ast}\).
            \end{definition}

            \begin{lemma}
            If \(i\in S^{\ast}\) and \(i \goto j\), then \(j \in S^{\ast}\)
            \end{lemma}

            \begin{proposition}
            \(\dgoto\) is an equivalence relation in \(S^{\ast}\), and:

            \[
            S^{\ast} = C_1 \cup C_2 \cup \ldots \cup C_r
            \]

            where for every \(i, j \in C_k\), it holds \(i \dgoto j\).
            \end{proposition}
        \end{column}
        \begin{column}{0.5\textwidth}
        Let $P_k$ be the transition matrix for the states in $C_k$ (which only 
        communicate with themselves); $Q_k$ the probabilities of going from 
        $S - S^{\ast}$ to $C_k$ (and never returning); and $R$ the probability
        of the non-essential states communicating with themselves. If on top of
        that we consider an all states are ordered and $C_k = [s_k]$ holds 
        \(s_{i} < s_{i + 1}\) for all \(i\).
        
        \[
            \left(\begin{array}{cccccccccc}
            \gcell P_1    &        &            &       &        &       &            &        & \\
                            & \rcell & \rcell     & \rcell&        &       &            &        & \\
                            & \rcell & \rcell P_2 & \rcell&        &       &            &        & \\
                            & \rcell & \rcell     & \rcell&        &       &            &        & \\
                            &        &            &       & \ddots &       &            &        & \\
                            &        &            &       &        &\bcell &   \bcell   & \bcell & \\
                            &        &            &       &        &\bcell & \bcell P_r & \bcell & \\
                            &        &            &       &        &\bcell &   \bcell   & \bcell & \\
                \rowcolor{gray!20}
                Q_1       &        &   Q_2      &       & \ldots &       &    Q_r     &        & R\\
            \end{array}\right)
        \]
        \end{column}

    \end{columns}
 \end{frame}
