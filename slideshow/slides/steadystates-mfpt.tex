\subsection{Steady states, mean first passage time}
 \begin{frame}\frametitle{Steady states, mean first passage time}
    \vspace{2em}
    \begin{columns}
     \begin{column}{0.5\textwidth}
        \begin{proposition}
        Given \(P_{m\times m}\) an stochastic matrix, then every steady state \(v\) 
        assigns \(0\) to the transient states.
        
        ~\\
        \label{prop:ss-transient-zero}
        \end{proposition}
        Each matrix \(P_1, \ldots, P_r\) from its canonic form is irreducible. 
        Therefore, by the fundamental theorem, each \(P_i\) has a unique steady vector 
        \(v^{(i)}\). 
        
        \begin{proposition}
        Given an stochastic matrix \(P\) written in its canonic form, then its space of 
        steady states is a convex hull given by:

        \[
        \bigg\{\sum_{i = 1}^r \alpha_i \widetilde{v}^{(i)}: 0 \le \alpha_i, \sum_{i = 1}^{r} \alpha_i = 1\bigg\}
        \]
        \end{proposition}
    \end{column}
    \begin{column}{0.5\textwidth}
        \begin{definition}
        We call the expected number of steps to reach a state \(j\) from initial state 
        \(i\) as mean first passage time \(m_{ij}\).
        
        We define the mean recurrence time for \(i\) as
        \(r_{i} := \expecti{\tau_{i}} = \sum_{n = 1}^{\infty} n f_{ii}^{(n)}\)
        \end{definition}
        
        \begin{proposition}
        Defining \(M = (m_{ij})\), \(D = (\delta_{ij} \cdot r_i)\), \(C = (1) \), then 
        it holds $M = PM + C - D$
        \end{proposition}
        
        \begin{proposition}
        Given an ergodic Markov chain, then \(r_i = w_i^{-1}\) where \(w\) is its unique steady state.
        \end{proposition}

        \begin{theorem}
        Given \(P\) an ergodic matrix, \(w\) its steady state, \(Z = (I - P + W)^{-1} = (z_{ij})\) and \(M = (m_{ij})\), it holds:

        \[ m_{ij} = \frac{z_{jj} - z_{ij}}{w_j} \]
        \end{theorem}
    \end{column}
    \end{columns}
\end{frame}
