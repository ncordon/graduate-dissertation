\begin{frame}\frametitle{Mathematical unit tests}
    \vspace{2.5em}
    
    Only testing on a couple of examples has the problem that it misses edge cases, and 
    it skips the point of testing a mathematical property \(\forall x P(x)\).
    
    \begin{enumerate}
    \item Generate random Markov chains of random dimnension, given both by-rows 
    stochastic matrices and by-column ones.
    \item Check that different implications, equivalences and characteristic 
    equation systems are verified.
    \end{enumerate}

    \begin{itemize}
    \item A state is absorbing iff it constitutes its own recurrent class.
    \item Recurrent and transient states form a partition of states.
    \item \(f_{i,i} < 1\) for a transient state (by defintion this must hold).
    \item \(f_{i,j} = 1\) for \(i,j\) in same recurrent class.
    \item \(f_{i,k} = 0\) if \(i\) is a recurrent state but \(k\) does not belong to 
    the same class.
    \item Union of recurrent (resp. transient) classes gives the recurrent (tesp. transient) states.
    \item Recurrent (resp. transient) classes are disjoint.
    \item Number of steady states should coincide with the number of recurrent classes.
    \item Steady states for a matrix are linearly independent (since they should be a base of the convex hull).
    \item Steady states are eigen probability vectors for the transition matrix.
    \end{itemize}

\end{frame}

\begin{frame} \frametitle{Mean recurrence time}

    The following algorithm has been added to the package (\(\algcomp{m^2}\) 
    complexity):

    \begin{algorithm}[H]
    \begin{algorithmic}[1]
    \REQUIRE $P_{m \times m}$ a stochastic matrix
    \NEWLINE
    \STATE{$(u_1 | \ldots | u_n) = \texttt{steadyStates}(P^t)$}
    \STATE{$r = (0, \ldots, 0)$}
    \FOR{Each steady state $u_i$}
        \FOR{$j = 1, \ldots, m$}
        \IF{$u_{ij} > 0$}
            \STATE{$r_j := u_{ij}^{-1}$}
        \ENDIF
        \ENDFOR
    \ENDFOR
    \RETURN{$\{(j, r_j): r_j > 0\}$}
    \end{algorithmic}
    \caption{\texttt{meanRecurrenceTime} algorithm}
    \label{alg:mrt}
    \end{algorithm}

\end{frame}

