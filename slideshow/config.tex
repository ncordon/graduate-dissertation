%------------------------
% Math libraries
%------------------------
\usepackage{amsthm}
\usepackage{amsmath}
\usepackage{tikz}
\usepackage{tikz-cd}
\usetikzlibrary{shapes,fit}
\usepackage{bussproofs}
\EnableBpAbbreviations{}
\usepackage{mathtools}
\usepackage{scalerel}
\usepackage{stmaryrd}


\usepackage{fix-cm}
\usepackage{dsfont}
\usepackage{ragged2e}
\usepackage{graphicx}
\definecolor{lightgray}{cmyk}{.30,0,0,.67} % define color using xcolor syntax
\definecolor{midgray}{gray}{0.40}
\usepackage{svg}
\usepackage{listings}
\usepackage{minted}
\usepackage{mdframed} % Add easy frames to paragraphs
\usepackage{lipsum} % For dummy text
\usepackage{xparse} % Add support for \NewDocumentEnvironment
\newmdenv[ % Define mdframe settings and store as leftrule
  linecolor=lightgray,
  topline=false,
  bottomline=false,
  rightline=false,
  skipabove=\topsep,
  skipbelow=\topsep
]{leftrule}

\addtobeamertemplate{block begin}{}{\justifying}
\apptocmd{\frame}{}{\justifying}{}
\apptocmd{\column}{}{\justifying}{}
\let\oldenumerate=\enumerate  
\renewenvironment{enumerate}{\justifying\oldenumerate}{\endlist}
\let\olditem=\item% 
\renewcommand{\item}{\olditem \justifying}


% *************************************************************************************************************
% Pseudocódigo de algoritmos
% *************************************************************************************************************
\usepackage{algorithm}
\usepackage{algorithmic}
\floatname{algorithm}{Algorithm}
\renewcommand{\listalgorithmname}{Algorithm list}
\renewcommand{\algorithmicrequire}{\textbf{Input }}
\renewcommand{\algorithmicensure}{\textbf{Output}}
\makeatletter
\def\NEWLINE{\STATE{}}
\makeatother

\usepackage{listings}
\renewcommand{\lstlistingname}{Código}
\definecolor{light-gray}{gray}{0.97}
\lstset{
  language=R,
  literate={<-}{{\boldmath{$\gets$}}}1,
  basicstyle=\small\linespread{1.3}\ttfamily,
  commentstyle=\ttfamily\color{OliveGreen},
  stringstyle=\color{Thistle},
  numbers=left,
  numberstyle=\scriptsize\ttfamily\color{RoyalBlue},
  stepnumber=1,
  numbersep=5pt,
  backgroundcolor=\color{light-gray},
  showspaces=false,
  showstringspaces=false,
  showtabs=false,
  %frame=single,
  tabsize=2,
  captionpos=b,
  breaklines=true,
  breakatwhitespace=false,
  %title=\lstname,
  escapeinside={},
  keywordstyle={},
  morekeywords={}
}



% *************************************************************************************************************
% Entornos
% *************************************************************************************************************
\usepackage{tabularx}
\newcolumntype{C}[1]{>{\centering\let\newline\\\arraybackslash\hspace{0pt}}m{#1}}
\usepackage{float}
\usepackage{wrapfig}

\NewDocumentEnvironment{example}{O{\textbf{Ejemplo:}}}
{\begin{leftrule}\noindent\textcolor{black}{#1}\par}
{\end{leftrule}}

\NewDocumentEnvironment{counterex}{O{\textbf{Contraejemplo:}}}
{\begin{leftrule}\noindent\textcolor{black}{#1}\par}
{\end{leftrule}}
% arg min
\DeclareMathOperator*{\argmin}{arg\,\min }
\DeclareMathOperator*{\argmax}{arg\,\max }

% norma || ||
\newcommand{\norm}[1]{||{#1}||}


%\defaultfontfeatures{Ligatures=TeX}

\newcommand{\img}[2]{
  \begin{center}
  \includegraphics[width=#2\textwidth]{#1}
  \end{center}
}

\newcommand{\imgcaption}[3]{
\begin{figure}[H]
  \begin{center}
  \includegraphics[width=#3\textwidth]{#1}
  \end{center}
  
  \caption[]{#2}
\end{figure}
}


%------------------------
% Theorem styles
%------------------------
\theoremstyle{definition}
\newtheorem{theorem}{Theorem}
\newtheorem{proposition}{Proposition}
\newtheorem{lemma}{Lemma}
\newtheorem{corollary}{Corollary}
\newtheorem{definition}{Definition}
\newtheorem{proofs}{Demonstration}
\theoremstyle{remark}
\newtheorem{remark}{Remark}
\newtheorem{exampleth}{Example}

%------------------------
% Macros
% ------------------------

% Latex shortcuts
\newcommand*\diff{\mathop{}\!\mathrm{d}}
\newcommand{\tildemc}{\widetilde{X}_n}
\newcommand{\tildex}{\widetilde{X}}
\newcommand{\tildeP}{\widetilde{P}}
\newcommand{\tildep}{\widetilde{p}}
\newcommand{\mc}{$\{X_n\}$}
\newcommand{\tildeprob}{\widetilde{p}}
\newcommand{\expecti}[1]{\mathbb{E}_i\left[{#1}\right]}
\newcommand{\expectj}[1]{\mathbb{E}_j\left[{#1}\right]}
\newcommand{\probi}[1]{P_i\left({#1}\right)}
\newcommand{\probj}[1]{P_j\left({#1}\right)}
\newcommand{\prob}[1]{P\left({#1}\right)}
\newcommand{\dimplies}{\Longleftrightarrow}
\newcommand{\goto}{\rightarrow}
\newcommand{\dgoto}{\leftrightarrow}

% Commands to divide a definition / equality into two parts
\newcommand{\twopartdef}[4]{
  \left\{\begin{array}{ll}
        #1 & \textrm{if } #2 \\
        #3 & \textrm{if } #4
  \end{array}\right.
}

\newcommand{\twopartsys}[4]{
  \left\{\begin{array}{lcl}
        #1 & = & #2 \\
        #3 & = & #4
  \end{array}\right.
}


\newcommand{\twopartdefow}[3]{
  \left\{\begin{array}{ll}
        #1 & \textrm{if } #2 \\
        #3 & \textrm{otherwise }
  \end{array}\right.
}

% and three parts
\newcommand{\threepartdef}[6]{
  \left\{\begin{array}{ll}
        #1 & \textrm{if } #2 \\
        #3 & \textrm{if } #4 \\
        #5 & \textrm{if } #6
  \end{array}\right.
}

\newcommand{\twoparteq}[2]{
  \left\{\begin{array}{l}
        #1\\
        #2
  \end{array}\right.
}

\newcommand{\threemat}[3]{
  \left(
    \begin{array}{ccc}
      #1 \\
      #2 \\
      #3
    \end{array}
  \right)
}

\newcommand{\algcomp}[1]{\mathcal{O}\left(#1\right)}
\newcommand{\card}[1]{\#{#1}}

\newcommand{\bigzero}{\mbox{\normalfont\Large\bfseries 0}}

% Number sets
\newcommand{\naturals}{\mathbb{N}}
\newcommand{\reals}{\mathbb{R}}
\newcommand{\complexes}{\mathbb{C}}
\newcommand{\posreals}{\mathbb{R}^{+}}

% bold 1 to represent a row / col full of 1s
\newcommand{\onecol}{\mathbf{1}}
\newcommand{\onerow}{\onecol}

\setbeamertemplate{enumerate item}{\roman{enumi}.}

% To be able to color matrices
\usepackage{blkarray}
\usepackage{colortbl}
\newcommand\x{\times}
\newcommand\rcell{\cellcolor{red!20}}
\newcommand\bcell{\cellcolor{blue!20}}
\newcommand\gcell{\cellcolor{green!20}}

\hyphenation{pers-pec-ti-ve me-thods exists exist ha-ving}
\usepackage{etoolbox}
