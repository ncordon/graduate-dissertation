% Setup visible during preview of math formulae

%------------------------
% Math libraries
%------------------------
\usepackage{amsthm}
\usepackage{amsmath}
\usepackage{tikz}
\usepackage{tikz-cd}
\usetikzlibrary{shapes,fit}
\usepackage{bussproofs}
\EnableBpAbbreviations{}
\usepackage{mathtools}
\usepackage{scalerel}
\usepackage{stmaryrd}

%------------------------
% Theorem styles
%------------------------
\theoremstyle{plain}
\newtheorem{theorem}{Theorem}
\newtheorem{proposition}{Proposition}
\newtheorem{lemma}{Lemma}
\newtheorem{corollary}{Corollary}
\theoremstyle{definition}
\newtheorem{definition}{Definition}
\newtheorem{proofs}{Demonstration}
\theoremstyle{remark}
\newtheorem{remark}{Remark}
\newtheorem{note}{Note}
\newtheorem{exampleth}{Example}

% To be able to color matrices
\usepackage{blkarray}
\usepackage{colortbl}
\newcommand\x{\times}
\newcommand\rcell{\cellcolor{red!20}}
\newcommand\bcell{\cellcolor{blue!20}}
\newcommand\gcell{\cellcolor{green!20}}

% ***************************************************************************************
% Environment for examples and counterexamples
% https://tex.stackexchange.com/questions/5223/command-for-argmin-or-argmax
% ***************************************************************************************

\usepackage{mdframed} % Add easy frames to paragraphs
\usepackage{lipsum} % For dummy text
\usepackage{xparse} % Add support for \NewDocumentEnvironment
\definecolor{lightgray}{cmyk}{.30,0,0,.67} % define color using xcolor syntax
\definecolor{midgray}{gray}{0.40}

\newmdenv[ % Define mdframe settings and store as leftrule
  linecolor=lightgray,
  topline=false,
  bottomline=false,
  rightline=false,
  skipabove=\topsep,
  skipbelow=\topsep
]{leftrule}

\NewDocumentEnvironment{example}{O{\textbf{Example:}}}
{\begin{leftrule}\noindent\textcolor{black}{#1}\par}
{\end{leftrule}}

\NewDocumentEnvironment{counterex}{O{\textbf{Counterexample:}}}
{\begin{leftrule}\noindent\textcolor{black}{#1}\par}
{\end{leftrule}}


\begingroup\makeatletter\@for\theoremstyle:=definition,remark,plain\do{\expandafter\g@addto@macro\csname th@\theoremstyle\endcsname{\addtolength\thm@preskip\parskip}}\endgroup

%------------------------
% Macros
% ------------------------

% Latex shortcuts
\newcommand*\diff{\mathop{}\!\mathrm{d}}
\newcommand{\tildemc}{\widetilde{X}_n}
\newcommand{\tildex}{\widetilde{X}}
\newcommand{\tildeP}{\widetilde{P}}
\newcommand{\tildep}{\widetilde{p}}
\newcommand{\mc}{$\{X_n\}$}
\newcommand{\tildeprob}{\widetilde{p}}
\newcommand{\expecti}[1]{\mathbb{E}_i\left[{#1}\right]}
\newcommand{\expectj}[1]{\mathbb{E}_j\left[{#1}\right]}
\newcommand{\probi}[1]{P_i\left({#1}\right)}
\newcommand{\probj}[1]{P_j\left({#1}\right)}
\newcommand{\prob}[1]{P\left({#1}\right)}
\newcommand{\dimplies}{\Longleftrightarrow}
\newcommand{\goto}{\rightarrow}
\newcommand{\dgoto}{\leftrightarrow}

% Commands to divide a definition into two parts
\newcommand{\twopartdef}[4]{
  \left\{\begin{array}{ll}
        #1 & \textrm{if } #2 \\
        #3 & \textrm{if } #4
  \end{array}\right.
}

% and three parts
\newcommand{\threepartdef}[6]{
  \left\{\begin{array}{ll}
        #1 & \textrm{if } #2 \\
        #3 & \textrm{if } #4 \\
        #5 & \textrm{if } #6
  \end{array}\right.
}

\newcommand{\twoparteq}[2]{
  \left\{\begin{array}{l}
        #1\\
        #2
  \end{array}\right.
}

\newcommand{\threemat}[3]{
  \left(
    \begin{array}{ccc}
      #1 \\
      #2 \\
      #3
    \end{array}
  \right)
}

\newcommand{\algcomp}[1]{\mathcal{O}\left(#1\right)}
\newcommand{\card}[1]{\#{#1}}