% General settings
% They will applied only when compiling to pdf.

% Geometry and spacing of paragraphs
\setcounter{secnumdepth}{0}
\usepackage{enumitem}
\setitemize{noitemsep,topsep=0pt,parsep=0pt,partopsep=0pt}
\setlist[enumerate]{topsep=0pt,itemsep=-1ex,partopsep=1ex,parsep=1ex}
\usepackage[top=1in, bottom=1.5in, left=1in, right=1in]{geometry}
\setlength\itemsep{0em}
\setlength{\parindent}{0pt}
\usepackage{parskip}

% Minted package for code listings
\usepackage{minted} \usemintedstyle{colorful}
\setminted{fontsize=\small}
\setminted[haskell]{linenos=false,fontsize=\small}
\renewcommand{\theFancyVerbLine}{\sffamily\textcolor[rgb]{0.5,0.5,1.0}{\oldstylenums{\arabic{FancyVerbLine}}}}

% Bibliography style
\bibliographystyle{config/custom-dinat}
\usepackage[sort&compress,square, comma, authoryear]{natbib}


% Pseudocode for algorithms
\usepackage{algorithm}
\usepackage{algorithmic}
%\floatname{algorithm}{Algoritmo}
%\renewcommand{\listalgorithmname}{Lista de algoritmos}
\renewcommand{\algorithmicrequire}{\textbf{Input:}}
\renewcommand{\algorithmicensure}{\textbf{Output:}}

\makeatletter
\def\NEWLINE{\STATE{}}
\makeatother

% Images
\usepackage{svg}

% Code listings
\usepackage{listings}
\renewcommand{\lstlistingname}{Code}
\definecolor{light-gray}{gray}{0.93}

\lstset{
  language=C++,
  basicstyle=\ttfamily,
  keywordstyle=\color{blue}\ttfamily,
  stringstyle=\color{red}\ttfamily,
  commentstyle=\color{green}\ttfamily,
  morecomment=[l][\color{magenta}]{\#}
}

\lstset{
  language=r,
  literate={<-}{{\boldmath{$\gets$}}}1,
  basicstyle=\small\linespread{1.3}\ttfamily,
  commentstyle=\ttfamily\color{OliveGreen},
  stringstyle=\color{Thistle},
  numbers=left,
  numberstyle=\scriptsize\ttfamily\color{RoyalBlue},
  stepnumber=1,
  numbersep=5pt,
  backgroundcolor=\color{light-gray},
  showspaces=false,
  showstringspaces=false,
  showtabs=false,
  %frame=single,
  tabsize=2,
  captionpos=b,
  breaklines=true,
  breakatwhitespace=false,
  %title=\lstname,
  escapeinside={},
  keywordstyle={},
  morekeywords={}
}

% Make enumerations as i., ii., iii.
\renewcommand{\theenumi}{\roman{enumi}}

% Hyphenation rules
\hyphenation{com-ple-xi-ty}